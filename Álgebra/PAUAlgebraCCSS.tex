\documentclass[12pt, a4paper]{amsart} 
\usepackage[utf8]{inputenc}
\usepackage[spanish]{babel}
\usepackage{anysize}
\usepackage{amsthm}
\usepackage{multicol}
\newcommand{\noun}[1]{\textsc{#1}}
\usepackage{amsmath}
\usepackage{amsfonts}
\usepackage{amssymb}
\usepackage{graphicx}
\usepackage{amscd}
\usepackage[colorlinks=true,linktocpage=true,pagebackref=true, citecolor=red,linkcolor=blue]{hyperref}
\usepackage{eurosym}
%opening

\newtheorem{teorema}{Teorema}[section]
\newtheorem{definicion}[teorema]{Definición}
\newtheorem{prop}[teorema]{Proposición}
\newtheorem{obs}[teorema]{Observación}
\newtheorem{cor}[teorema]{Corolario}
\newtheorem{ejem}[teorema]{Ejemplo}
\newtheorem{ejems}[teorema]{Ejemplos}
\newtheorem{ejer}{Ejercicio}

\marginsize{2cm}{2cm}{1cm}{1cm}

\begin{document}


\title{Ejercicios de álgebra PAU - Matemáticas aplicadas a las ciencias sociales II}
\maketitle
\date{}
\thispagestyle{empty}

\begin{ejer}\em (2021-2022)\\
Sea $a\in\mathbb{R}.$ Considere las matrices
\[
A=\begin{pmatrix}
-a & 1 & -2\\
0 & -1 & 1\\
a & a & -1
\end{pmatrix}\hspace*{1cm} 
B=\begin{pmatrix}
1 & 2 & 1\\
0 & -1 & 0\\
1 & 1 & 2
\end{pmatrix}\hspace*{1cm}
Y=\begin{pmatrix}
1\\
0\\
2
\end{pmatrix}
\]
a) Determine los valores de $a$ para que $A$ tenga inversa.\\
b) Calcule los valores de $a$ para que la solución del sistema $(A - B)X = Y$ sea
\[\begin{pmatrix}
0\\
-1\\
0
\end{pmatrix}.\]
\end{ejer}

\begin{ejer}\em (2021-2022)\\
Considere el sistema de ecuaciones lineales dependiente del parámetro  $a\in\mathbb{R}$
\begin{equation*}
\left \{ \begin{matrix} x+2y+z=2
\\ ax-z=0
\\ay+z=a \end{matrix}\right. 
\end{equation*}
a) Determine a para que el sistema NO sea compatible determinado.\\
b) Resuelva el sistema para $a = 2.$
\end{ejer}

\begin{ejer}\em (2021-2022)\\
Sea $a\in\mathbb{R}.$ Considere las matrices
\[
A=\begin{pmatrix}
-a & 1 & 1\\
0 & -1 & 1\\
a & 1 & 1
\end{pmatrix}\hspace*{1cm} 
B=\begin{pmatrix}
1 & 1 & 1\\
-1 & 1 & 0\\
1 & 1 & 2
\end{pmatrix}\hspace*{1cm}
X=\begin{pmatrix}
x\\
y\\
z
\end{pmatrix}
Y=\begin{pmatrix}
1\\
0\\
2
\end{pmatrix}
\]
a) Determine los valores de $a$ para que $A$ tenga inversa.\\
b) Calcule, para $a=1,$ la solución del sistema $(A - B)X = Y.$ 
\end{ejer}

\begin{ejer}\em (2021-2022)\\
Considere el sistema de ecuaciones lineales dependiente del parámetro  $a\in\mathbb{R}$
\begin{equation*}
\left \{ \begin{matrix} x+ay+z=2
\\ x-az=0
\\x+y+z=2 \end{matrix}\right. 
\end{equation*}
a) Discuta la compatibilidad del sistema para los diferentes valores de $a.$\\
b) Resuelva el sistema para $a = 0.$
\end{ejer}

\begin{ejer}\em (2021-2022)\\
Sea $a\in\mathbb{R}.$ Considere las matrices
\[
A=\begin{pmatrix}
0 & 1 & 0\\
1 & 0 & 0\\
0 & 1 & 0
\end{pmatrix}\]
y $B = A + aI,$ donde $I$ es la matriz identidad de orden 3 y $a$ es un número real.\\
a) Calcule $A(A^2 - A^4 )$\\
b) Calcule los valores de $a$ para que las matrices $B$ y $AB$ sean invertibles.
\end{ejer}

\begin{ejer}\em (2021-2022)\\
Considere el sistema de ecuaciones lineales dependiente del parámetro  $a\in\mathbb{R}$
\begin{equation*}
\left \{ \begin{matrix} 2x+z=1
\\ ax-y+z=0
\\ay+z=a+1 \end{matrix}\right. 
\end{equation*}
a) Discuta la compatibilidad del sistema para los diferentes valores de $a.$\\
b) Resuelva el sistema para $a = 0.$
\end{ejer}

\begin{ejer}\em (2021-2022)\\
Se considera la matriz
\[A=\begin{pmatrix}
2 & 0 & 1
\\a & -1 & 1
\\ 0 & a & 1 \end{pmatrix}\]
a) Determine losvalores del parámetro real $a$ para los cuales la matriz $A$ es invertible.\\
b) Calcule $A^{-1}$ para $a = 1.$
\end{ejer}

\begin{ejer}\em (2021-2022)\\
Se considera el sistema de ecuaciones lineales dependiente del parámetro $a\in\mathbb{R}:$
\begin{equation*}
\left \{ \begin{matrix} x+ay+z=a
\\ ax-y-az=0
\\x+y+z=1 \end{matrix}\right. 
\end{equation*}
a) Discuta la compatibilidad del sistema para los diferentes valores de $a.$\\
b) Resuelva el sistema para $a = 2.$
\end{ejer}

\begin{ejer}\em (2020-2021)\\
Se considera el sistema de ecuaciones lineales dependiente del parámetro real $a:$
\[
\left \{ \begin{matrix}
x+2ay+z=0
\\ -x-ay=1
\\  -y-z=-a
\end{matrix}  \right.
\]
a) Discuta el sistema en función de los valores del parámetro $a.$\\
b) Resuelva el sistema para $a = 3.$

\end{ejer}

\begin{ejer}\em (2020-2021)\\
Se consideran las matrices $A=\begin{pmatrix}
a & 1 & 1
\\ -1 & 2 & 0
\\ 0 & -a & -1
\end{pmatrix} \text{ y } B=\begin{pmatrix}
-2
\\ 1
\\ -1
\end{pmatrix}$\\
a) Calcule los valores del parámetro real $a$ para los cuales la matriz $A$ tiene inversa.\\
b) Para $a = 2$ calcule, si existe, la matriz $X$ que satisface $AX = B .$
\end{ejer}

\begin{ejer}\em (2020-2021)\\
Se considera el sistema de ecuaciones lineales dependiente del parámetro real $a:$
\[
\left \{ \begin{matrix}
x+y-z=-1
\\ x-y+a^2z=3
\\  2x-y+z=4
\end{matrix}  \right.
\]
a) Discuta el sistema en función de los valores del parámetro $a.$\\
b) Resuelva el sistema para $a = 1.$
\end{ejer}

\begin{ejer}\em (2020-2021)\\
Se considera la matriz $A$
\[
A=\begin{pmatrix}
a & 0 & 1
\\ 0 & b & 0
\\ 1 & 0 & a
\end{pmatrix}
\]
a) Determine los valores de los parámetros reales $a$ y $b$ para los que $A = A^{-1}.$\\
b) Para $a = b = 2,$ calcule la matriz inversa de $A.$
\end{ejer}

\begin{ejer}\em (2019-2020)\\
Se considera el sistema de ecuaciones lineales dependiente del parámetro real $a\in\mathbb{R}:$
\[
\left. \begin{matrix}
x-ay=1
\\ ax-4y-z=2
\\  2x+ay-z=a-4
\end{matrix}  \right \}
\]
a) Discuta el sistema para los diferentes valores de $a .$\\
b) Resuelva el sistema para $a = 3.$
\end{ejer}

\begin{ejer}\em (2019-2020)\\
Dada la matriz $A=\begin{pmatrix}
2 & 5a
\\ a & 3\end{pmatrix}$ con $a\in\mathbb{R}.$\\
a) Determine los valores del parámetro $a$ para los que se verifica la igualdad $A^2 - 5A = -I ,$ donde $I$ es la matriz identidad.\\
b) Calcule $A^{-1}$ para $a = -1 .$
\end{ejer}

\begin{ejer}\em (2019-2020)\\
Se considera la matriz $A$ dada por
\[A=\begin{pmatrix}
3 & 1 & 2
\\ 0 & m & 0
\\ 1 & -1 & 2
\end{pmatrix}\]
a) Calcule el valor del parámetro real $m$ para que $A^2 - 5A = -4I ,$ siendo $I$ la matriz identidad.\\
b) Para $m = 1 ,$ indique si la matriz $A$ es invertible y, en caso afirmativo, calcule su inversa.
\end{ejer}

\begin{ejer}\em (2019-2020)\\
Se considera el sistema de ecuaciones lineales dependiente del parámetro real $a :$
\[
\left. \begin{matrix}
x+ay=0
\\ x+2z=0
\\  x+ay+(a+1)z=a
\end{matrix}  \right \}
\]
a) Discuta el sistema en función de los valores del parámetro $a .$\\
b) Resuelva el sistema para $a = 0 .$
\end{ejer}

\begin{ejer}\em (2018-2019)\\
Se considera la matriz
\[A=\begin{pmatrix}
3 & 8 & 10
\\ 2 & 1 & 2
\\ 4 & 3 & 6
\end{pmatrix}\]
y la matriz $B$ es tal que
\[
(AB)^{-1}=\frac{1}{2}\begin{pmatrix}
0 & 3 & -1
\\ 0 & -1 & 1
\\ 2 & -3 & -3
\end{pmatrix}
\]
a) Calcúlese $A^{-1} .$\\
b) Calcúlese $B^{-1}.$
\end{ejer}

\begin{ejer}\em (2018-2019)\\
Sean las matrices
\[A=\begin{pmatrix}
a & 4 & 2 
\\ 1 & a & 0
\\ 1 & 2 & 1
\end{pmatrix}, \hspace*{1cm} B=\begin{pmatrix}
9
\\ 1
\\3
\end{pmatrix}\]
a) Calcúlense los valores de a para los cuales la matriz $A$ no tiene matriz inversa.\\
b) Para $a = 3,$ calcúlese la matriz inversa de $A$ y resuélvase la ecuación matricial $A \cdot X = B.$
\end{ejer}

\begin{ejer}\em (2018-2019)\\
Se considera el siguiente sistema de ecuaciones lineales dependiente de un parámetro real $m:$
\[
\left. \begin{matrix}
-x+y+z=0
\\ x+my-z=0
\\  x-y-mz=0
\end{matrix}  \right \}
\]
a) Determínense los valores del parámetro real $m$ para que el sistema tenga soluciones diferentes a la solución trivial $x = y = z = 0.$\\
b) Resuélvase el sistema para $m = 1.$
\end{ejer}

\begin{ejer}\em (2018-2019)\\
Se consideran las siguientes matrices
\[A=\begin{pmatrix}
k & 1 & 2 
\\ 1 & 4 & 3
\\ 0 & 0 & 7
\end{pmatrix} \hspace*{1cm} B=\begin{pmatrix}
1 & 0 & 1
\\ 0 & 1 & 0
\\4 & 0 & 3
\end{pmatrix} \hspace*{1cm} C=\begin{pmatrix}
1 & 1
\\ 0 & -1
\\ 1 & 0
\end{pmatrix}\]
a) Obténgase el valor de la constante $k$ para que el determinante de la matriz $A - 2B$ sea nulo.\\
b) Determínese si las matrices $C$ y $(C^t \cdot C),$ donde $C^t$ denota la matriz traspuesta de $C,$ son invertibles. En caso afirmativo, calcúlense las inversas.
\end{ejer}

\begin{ejer}\em (2017-2018)\\
Se considera el sistema de ecuaciones dependiente del parámetro real a:
\[
\left. \begin{matrix}
x+3y+z=a
\\ 2x+ay-6z=8
\\  x-3y-5z=4
\end{matrix}  \right \}
\]
a) Discútase en función de los valores del parámetro $a.$\\
b) Resuélvase para $a = 4.$
\end{ejer}

\begin{ejer}\em (2017-2018)\\
Considérense las matrices $A=\begin{pmatrix}
0 & 1 
\\ 1 & 0
\\ 0 & 1
\end{pmatrix} \text{ y } B=\begin{pmatrix}
3 
\\ 2
\\ 3
\end{pmatrix}$\\
a) Calcúlese la matriz $[(A\cdot A^t)^2-2A\cdot A^t]^{11}$\\
b) Determínense el número de filas y columnas de la matriz $X$ que verifica que $X\cdot A^t = B^t.$ Justifíquese si $A^t$ es una matriz invertible y calcúlese la matriz $X .$\\
Nota: $M^t$ denota la matriz traspuesta de la matriz $M.$
\end{ejer}

\begin{ejer}\em (2017-2018)\\
Se considera el sistema de ecuaciones dependiente del parámetro real a:
\[
\left. \begin{matrix}
x+ay+z=1
\\ ax+y+(a-1)z = a
\\  x+y+z=a+1
\end{matrix}  \right \}
\]
a) Discútase en función de los valores del parámetro $a.$\\
b) Resuélvase para $a = 3.$
\end{ejer}

\begin{ejer}\em (2017-2018)\\
Considérense las matrices $A=\begin{pmatrix}
3 & 1 
\\ 8 & 3
\end{pmatrix} \hspace*{1cm} B=\begin{pmatrix}
3 & -1
\\ -8 & 3
\end{pmatrix}$\\
a) Compruébese que $B$ es la matriz inversa de $A.$\\
b) Calculése la matriz $X$ tal que $A \cdot X = B.$
\end{ejer}

\begin{ejer}\em (2016-2017)\\
Considérense las matrices:
\[A=\begin{pmatrix}
1 & -2 
\\ -1 & 1
\end{pmatrix} \hspace*{1cm} B=\begin{pmatrix}
1 & 3
\\ 2 & -1
\end{pmatrix} \hspace*{1cm} C=\begin{pmatrix}
-1 & 0
\\ 3 & 1
\end{pmatrix}\]
a) Determínese la matriz $C^{40}.$\\
b) Calcúlese la matriz $X$ que verifica
\[X\cdot A + 3B=C\]
\end{ejer}

\begin{ejer}\em (2016-2017)\\
Considérese el sistema de ecuaciones dependiente del parámetro real $a:$
\[
\left \{ \begin{matrix}
x -2y -z=-2
\\ -2x -ax =2
\\  y+az=-2
\end{matrix}  \right.
\]
a) Discútase en función de los valores del parámetro $a.$\\
b) Resuélvase para $a = 4.$
\end{ejer}

\begin{ejer}\em (2016-2017)\\
Considérese el sistema de ecuaciones dependiente del parámetro real $a:$
\[
\left \{ \begin{matrix}
x -ay+2z=0
\\ ax-4y -4z=0
\\ (2-a)x+3y-2z=0
\end{matrix}  \right.
\]
a) Discútase en función de los valores del parámetro $a.$\\
b) Resuélvase para $a = 3.$
\end{ejer}

\begin{ejer}\em (2016-2017)\\
Considérense las matrices:
\[
A=\begin{pmatrix}
1 & 2 & -k
\\ 1 & -2 & 1
\\ k & 2 & -1
\end{pmatrix} \hspace*{1cm} B=\begin{pmatrix}
1 & 1 & 1
\\ 0 & 2 & 2
\\0 & 0 & 3
\end{pmatrix}\]
a) Discútase para qué valores del parámetro real $k$ la matriz $A$ tiene matriz inversa.\\
b) Determínese para $k = 0$ la matriz $X$ que verifica la ecuación $A \cdot X = B.$
\end{ejer}

\begin{ejer}\em (2015-2016)\\
Se considera la matriz $A=\begin{pmatrix}
k & -1 & 0
\\ -7 & k & k
\\ -1 & -1 & k\end{pmatrix}$\\
a) Estúdiese para qué valores del parámetro real $k$ la matriz $A$ tiene inversa.\\
b) Determínese, para $k = 1,$ la matriz $X$ tal que $X\cdot A = Id.$\\
Nota: $Id$ denota la matriz identidad de tamaño $3 \times 3.$
\end{ejer}

\begin{ejer}\em (2015-2016)\\
Se considera el sistema de ecuaciones dependientes del parámetro real $a:$
\[
\left. \begin{matrix}
(a-1)x+y+z=1
\\ x+(a-1)y+(a-1)z=1
\\ x+az=1
\end{matrix}  \right \}
\]
a) Discútase el sistema según los valores de $a.$\\
b) Resuélvase el sistema para $a = 3.$
\end{ejer}

\begin{ejer}\em (2015-2016)\\
Se considera el sistema de ecuaciones lineales:
\[
\left \{ \begin{matrix}
x+2y+z=1
\\x+2y+3z=0
\\ x+ay+2z=0
\end{matrix}\right.
\]
a) Discútase para los diferentes valores del parámetro $a\in\mathbb{R}.$\\
b) Resuélvase para $a = 0.$
\end{ejer}

\begin{ejer}\em (2015-2016)\\
Considérense las matrices
\[
A=\begin{pmatrix}
3 & 2 & 2
\\ 1 & 7 & 4
\\ 2 & 4 & 2
\end{pmatrix} \hspace*{1cm} B=\begin{pmatrix}
2 & 1
\\ 5 & 3
\\0 & 1
\end{pmatrix} \hspace*{1cm} C=\begin{pmatrix}
2 & 4 & 8
\\ 0 & 1 & 1
\\0 & 0 & 1
\end{pmatrix}
\]
a) Calcúlese el determinante de la matriz
\[
A\cdot C \cdot C^T\cdot A^{-1}
\]
b) Calcúlese la matriz $M = A \cdot B.$ ¿Existe $M^{-1}$?\\
Nota: $C^T$ denota la matriz traspuesta de la matriz $C.$
\end{ejer}

\begin{ejer}\em (2014-2015)\\
Considérese el sistema de ecuaciones dependiente del parámetro real $a:$\\
\[
\left \{ \begin{matrix}
x+y+az=a+1
\\ax+y+z=1
\\ x+ay+az=a
\end{matrix}\right.
\]
a) Discútase el sistema en función de los valores de $a.$\\
b) Resuélvase el sistema para $a=2.$
\end{ejer}

\begin{ejer}\em  (2014-2015)\\
Se consideran las matrices
\[
A=\begin{pmatrix}
3 & 1
\\ -6 & -2
\end{pmatrix} \text{ y } B=\begin{pmatrix}
1 & -3
\\ -1 & 2
\end{pmatrix}
\]
a) Calcúlese $A^{15}$ e indíquese si la matriz $A$ tiene inversa.\\
b) Calcúlese el determinante de la matriz $(B\cdot A^tB^{-1}-2\cdot Id)^3.$\\
\em Nota: $A^t$ denota la matriz traspuesta de $A.$ $Id$ es la matriz identidad de orden 2.
\end{ejer}

\begin{ejer}\em  (2014-2015)\\
Sea la matriz
\[
A=\begin{pmatrix}
2 & 2 & 0
\\ 0 & 3 & 2
\\ -1 & k & 2
\end{pmatrix}
\]
a) Estúdiese el rango de $A$ según los valores del parámetro real $k.$\\
b) Calcúlese, si existe, la matriz inversa de $A$ para $k=3.$
\end{ejer}

\begin{ejer}\em  (2014-2015)\\
Se considera el sistema de ecuaciones dependiente del parámetro real $a:$
\[
\left \{ \begin{matrix}
3x+y-z=8
\\ 2x+az=3
\\ x+y+z=2
\end{matrix}\right.
\]
a) Discútase en función de los valores del parámetro $a.$\\
b) Resuélvase para $a=1.$
\end{ejer}

\begin{ejer}\em  (2013-2014)\\
Considérese la matriz 
\[A=\begin{pmatrix}
1 & 0 \\ 0 & 0\\ 0 & 1
\end{pmatrix}
\]
a) Calcúlese $(A\cdot A^t)^{200}.$\\
b) Calcúlese $(A\cdot A^t - 3I)^{-1}$\\
\em Nota: $A^t$ denota a la traspuesta de la matriz $A.$ I es la matriz identidad de orden 3.
\end{ejer}

\begin{ejer}\em  (2013-2014)\\
Considérese el siguiente sistema de ecuaciones dependiente del parámetro real $\lambda :$
\[
\left \{ \begin{matrix}
2x-\lambda y+z=-\lambda
\\ 4x-2\lambda y +2z=\lambda-3
\end{matrix}\right.
\]
a) Determínense los valores del parámetro real $\lambda$ que hacen que el sistema sea incompatible.\\
b) Resuélvase el sistema para $\lambda=1.$
\end{ejer}

\begin{ejer}\em  (2013-2014)\\
Se considera el sistema de ecuaciones dependiente del parámetro real $a:$\\
\[
\left. \begin{matrix}
x+y+az=2
\\ 3x+4y+2z=a
\\ 2x+3y-z=-1
\end{matrix}  \right \}
\]
a) Discútase el sistema según los diferentes valores de $a.$\\
b) Resuélvase el sistema en el caso $a=-1.$
\end{ejer}

\begin{ejer}\em  (2013-2014)\\
Sean las matrices $A=\begin{pmatrix}
2 & 1\\ -1 & 0\\ 1 & -2
\end{pmatrix}\text{ y } B=\begin{pmatrix}
3 & 1\\ 0 & 2 \\ -1 & 0
\end{pmatrix}$\\
a) Calcúlese $(A^tB)^{-1},$ donde $A^t$ denota a la matriz traspuesta de la matriz $A.$\\
b) Resuélvase la ecuación matricial $A\cdot \begin{pmatrix}
x\\y
\end{pmatrix}=\begin{pmatrix}
0\\-1\\ 5
\end{pmatrix}.$
\end{ejer}

\begin{ejer}\em (2012-2013)\\
Se considera el siguiente sistema lineal de ecuaciones, dependiente del parámetro real $a:$
\begin{equation*}
\left \{ \begin{matrix} ax-2y=2
\\ 3x-y-z=-1
\\ x+3y+z=1 \end{matrix}\right. 
\end{equation*}
a) Discútase en función de los valores del parámetro $a\in\mathbb{R}.$\\
b) Resuélvase para $a=1.$
\end{ejer}

\begin{ejer}\em (2012-2013)\\
Se consideran las matrices $A=\begin{pmatrix}
0 & 2
\\ 3 & 0 \end{pmatrix}$ y $B=\begin{pmatrix}
-3 & 8
\\ 3 & -5 \end{pmatrix}.$\\
a) Calcúlese la matriz inversa de $A.$\\
b) Resuélvase la ecuación matricial $A\cdot X=B-I,$ donde $I$ es la matriz identidad.
\end{ejer}


\begin{ejer}\em (2012-2013)\\
Dada la matriz \begin{equation*}
A=\begin{pmatrix}
3 & 2 & 0
\\ 1 & 0 & -1
\\ 1 & 1 & 1
\end{pmatrix}.
\end{equation*}
a) Calcúlese $A^{-1}.$\\
b) Resuélvase el sistema de ecuaciones dado por: \begin{equation*}
A\cdot \begin{pmatrix}
x\\ y\\ z
\end{pmatrix}= \begin{pmatrix}
1\\ 0\\ 1
\end{pmatrix}.
\end{equation*}
\end{ejer}

\begin{ejer}\em  (2012-2013)\\
Se considera el siguiente sistema lineal de ecuaciones, dependiente del parámetro real $k:$
\begin{equation*}
\left \{ \begin{matrix} kx+y=0
\\ x+ky-2z=1
\\ kx-3y+kz=0 \end{matrix}\right. 
\end{equation*}
a) Discútase el sistema para los diferentes valores de $k.$\\
b) Resuélvase el sistema para $k=1.$
\end{ejer}

\begin{ejer}\em (2011-2012)\\
Se considera el siguiente sistema lineal de ecuaciones, dependiente del parámetro real $a:$
\begin{equation*}
\left \{ \begin{matrix} x+ay-7z=4a-1
\\ x+(1+a)y -(a+6)z =3a+1
\\ ay-6z=3a-2 \end{matrix}\right. 
\end{equation*}
a) Discútase el sistema para los diferentes valores de $a.$\\
b) Resuélvase el sistema en el caso en que tenga infinitas soluciones.\\
c) Resuélvase el sistema para $a=-3.$
\end{ejer}

\begin{ejer}\em (2011-2012)\\
Un estadio de fútbol con capacidad para 72000 espectadores está lleno durante la celebración de un partido entre los equipos A y B. Unos espectadores son socios del equipo A, otros lo son del equipo B, y el resto no son socios de ninguno de los equipos que están jugando. A través de la venta de localidades sabemos lo siguiente:\\
a) No hay espectadores que sean socios de ambos equipos simultáneamente.\\
b) Por cada 13 socios de alguno de los dos equipos hay 3 espectadores que no son socios.\\
c) Los socios del equipo B superan en 6500 a los socios del equipo A.\\
¿Cuántos socios de cada equipo hay en el estadio viendo el partido?
\end{ejer}


\begin{ejer}\em (2010-2011)\\
Se considera el sistema lineal de ecuaciones dependiente del parámetro real $a:$
\begin{equation*}
\left \{ \begin{matrix} ax+y+z=a
\\ ay+z=1
\\ ax+y+az=a \end{matrix}\right. 
\end{equation*}
a) Discútase el sistema para los distintos valores de $a.$\\
b) Resuélvase el sistema en el caso en que tenga infinitas soluciones.\\
c) Resuélvase el sistema para $a=3.$
\end{ejer}

\begin{ejer}\em (2010-2011)\\
Se consideran las matrices:
\begin{equation*}
A=\begin{pmatrix}
-1 & 0 & 1
\\ 3 & k & 0
\\ -k & 1 & 4 \end{pmatrix};
B=\begin{pmatrix}
3 & 1
\\ 0 & 3
\\ 2 & 0 \end{pmatrix}
\end{equation*}
a) Calcúlense los valores de $k$ para los cuales la matriz $A$ no es invertible.\\
b) Para $k=0,$ calcúlese la matriz inversa $A^{-1}.$\\
c) Para $k=0,$ resuélvase la ecuación matricial $AX=B.$
\end{ejer}

\begin{ejer}\em (2010-2011)\\
Se consideran las matrices:
\[A=\begin{pmatrix}
0 & 0 \\
1 & 1
\end{pmatrix};\hspace*{1cm} B=\begin{pmatrix}
1 & a\\
1 & b
\end{pmatrix};\hspace*{1cm}  I=\begin{pmatrix}
1 & 0\\
0 & 1
\end{pmatrix};\hspace*{1cm} O=\begin{pmatrix}
0 & 0\\
0 & =
\end{pmatrix}\]
a) Calcúlense $a,b$ para que se verifique la igualdad $AB=BA.$\\
b) Calcúlense $c,d$ para que se verifique la igualdad $A^2+cA+dI=O.$\\
c) Calcúlense todas las soluciones del sistema lineal
\[(A-I)\begin{pmatrix}
x\\y
\end{pmatrix}=\begin{pmatrix}
0\\0
\end{pmatrix}\]
\end{ejer}

\begin{ejer}\em (2009-2010)\\
Se considera el siguiente sistema lineal de ecuaciones, dependiente del parámetro real $k:$
\begin{equation*}
\left \{ \begin{matrix} x-y+kz=1
\\ 2x-ky+z=2
\\ x-y-z=k-1 \end{matrix}\right. 
\end{equation*}
a) Discútase el sistema para los distintos valores de $k.$\\
b) Resuélvase el sistema para el valor de $k$ para el cual tiene infinitas soluciones.\\
c) Resuélvase el sistema para $k=3.$
\end{ejer}

\begin{ejer}\em (2009-2010)\\
Se considera el siguiente sistema lineal de ecuaciones, dependiente del parámetro real $k:$
\begin{equation*}
\left \{ \begin{matrix} kx-2y+7z=8
\\ x-y+kz=2
\\ -x+y+z=2 \end{matrix}\right. 
\end{equation*}
a) Discútase el sistema para los distintos valores de $k.$\\
b) Resuélvase el sistema en el caso en que tenga infinitas soluciones.\\
c) Resuélvase el sistema para $k=0.$
\end{ejer}

\begin{ejer}\em (2009-2010)\\
Se considera el siguiente sistema lineal de ecuaciones dependiente del parámetro real $a:$
\begin{equation*}
\begin{pmatrix}
1
\\ 2
\\ 1 \end{pmatrix} x +
\begin{pmatrix}
1 & -1
\\ -3 & 2
\\ -4 & a \end{pmatrix}
\begin{pmatrix}
y
\\ z \end{pmatrix}=
\begin{pmatrix}
1
\\ 22
\\ 7a \end{pmatrix}
\end{equation*}
a) Discútase el sistema para los distintos valores del parámetro $a.$\\
b) Resuélvase el sistema para el valor de $a$ para el cual el sistema tiene infinitas soluciones.\\
c) Resuélvase el sistema para $a=0.$
\end{ejer}

\begin{ejer}\em (2009-2010)\\
Se consideran las matrices:
\begin{equation*}
A=\begin{pmatrix}
a-3 & 2 & -1
\\2 & a & 2
\\ 2a & 2(a+1) & a+1 \end{pmatrix};
X=\begin{pmatrix}
x
\\ y 
\\ z \end{pmatrix};
O=\begin{pmatrix}
0
\\ 0
\\0 \end{pmatrix}
\end{equation*}
a) Calcúlense los valores de $a$ para los cuales no existe la matriz inversa $A^{-1}.$\\
b) Para $a=-1,$ calcúlese la matriz inversa  $A^{-1}.$\\
c) Para $a=0,$ calcúlense todas las soluciones del sistema lineal $AX=O.$
\end{ejer}


\begin{ejer}\em (2008-2009)\\
Se considera el siguiente sistema lineal de ecuaciones, dependiente del parámetro real $k:$
\begin{equation*}
\left \{ \begin{matrix} x+y+kz=4
\\ 2x-y+2z=5
\\ -x+3y-z=0 \end{matrix}\right. 
\end{equation*}
a) Discútase el sistema para los distintos valores de $k.$\\
b) Resuélvase el sistema en el caso en que tenga infinitas soluciones.\\
c) Resuélvase el sistema para $k=0.$
\end{ejer}

\begin{ejer}\em  (2008-2009)\\
Se considera el siguiente sistema lineal de ecuaciones, dependiente del parámetro real $k:$
\begin{equation*}
\left \{ \begin{matrix} x+y+z=3
\\ x+ky+z=3
\\ kx-3z=6 \end{matrix}\right. 
\end{equation*}
a) Discútase el sistema para los distintos valores de $k.$\\
b) Resuélvase el sistema en el caso en que tenga infinitas soluciones.\\
c) Resuélvase el sistema para $k=3.$
\end{ejer}

\begin{ejer}\em (2007-2008)\\%
Un agricultor tiene repartidas sus 10 hectáreas de terreno en barbecho, cultivo de trigo y cultivo de cebada. La superficie dedicada al trigo ocupa 2 hectáreas más que la dedicada a la cebada, mientras que en barbecho tiene 6 hectáreas menos que la superficie totoal dedicada al cultivo de trigo y cebada. ¿Cuántas hectáreas tiene dedicadas a cada uno de los cultivos y cuántas están en barbecho?
\end{ejer}

\begin{ejer}\em  (2007-2008)\\%
Una empresa instala casas prefabricadas de tres tipos A, B Y C. Cada casa de tipo A necesita 10 horas de albañilería, 2 de fontanería y 2 de electricista. Cada casa de tipo B necesita 15 horas de albañilería, 4 de fontanería y 3 de electricista. Cada casa de tipo C necesita 20 horas de albañilería, 6 de fontanería y 5 de electricista. La empresa emplea exactamente 270 horas de trabajo al mes de albañilería, 68 de fontanería y 58 de electricista. ¿Cuántas casas de cada tipo instala la empresa en un mes?
\end{ejer}

\begin{ejer}\em (2006-2007)\\
Se considera el sistema lineal de ecuaciones, dependiente del parámetro real $a:$
\begin{equation*}
\left \{ \begin{matrix} x-2y+z=0
\\ 3x+2y-2z=3
\\ 2x+2y+az=8 \end{matrix}\right. 
\end{equation*}
a) Discutir el sistema para los distintos valores de $a.$\\
b) Resolver el sistema para $a=4.$
\end{ejer}

\begin{ejer}\em (2006-2007)\\
Dado el sistema lineal de ecuaciones, dependiente del parámetro real $a:$
\begin{equation*}
\left \{ \begin{matrix} x+ay+z=1
\\ 2y+az=2
\\x+y+z=1 \end{matrix}\right. 
\end{equation*}
a) Discutir el sistema para los distintos valores de $a.$\\
b) Resolver el sistema para $a=3$ y $a=1.$
\end{ejer}


\end{document}

\begin{ejer}\em 

\end{ejer}


\begin{ejer}\em %
Se considera el sistema lineal de ecuaciones dependiente del parámetro real $a:$
\begin{equation*}
\left \{ \begin{matrix} ax-2y\ \ \ =2
\\ 3x-y-z=-1
\\ x+3y+z=1 \end{matrix}\right. 
\end{equation*}
a) Discútase en función de los valores del parámetro $a\in\mathbb{R}.$\\
b) Resuélvase para $a=1.$
\end{ejer}

\begin{ejer}\em 
Se considera el sistema de ecuaciones lineales, dependiente del parámetro $k:$
\begin{equation*}
\left \{ \begin{matrix} kx+y\ \ \ =0
\\ x+ky-2z=1
\\ kx-3y+kz=0 \end{matrix}\right. 
\end{equation*}
a) Discútase el sistema según los diferentes valores de $k.$\\
b) Resuélvase el sistema para $k=1.$
\end{ejer}


\begin{ejer}\em %
Se considera el siguiente sistema lineal de ecuaciones, dependiente del parámetro real $k:$
\begin{equation*}
\left \{ \begin{matrix} x+y+z=2
\\ x+ky+2z=5
\\ kx+y+z=1 \end{matrix}\right. 
\end{equation*}
a) Discútase el sistema para los distintos valores de $k.$\\
b) Resuélvase el sistema para $k=0.$\\
c) Resuélvase el sistema para $k=2.$
\end{ejer}

\begin{ejer}\em
Dada la matriz $$A=\begin{pmatrix}
3 & 2 & 0
\\ 1 & 0 &-1
\\ 1 & 1 & 1
\end{pmatrix}.
$$
a) Calcúlese $A^{-1}.$\\
b) Resuélvase el sistema de ecuaciones dado por:
\begin{equation*}
A\cdot \begin{pmatrix}
x
\\ y
\\ z \end{pmatrix}=\begin{pmatrix}
1
\\ 0 
\\ 1 \end{pmatrix}
\end{equation*}
\end{ejer}

\begin{ejer}\em  Modelo 2016\\%
Considérese la matriz $A=\begin{pmatrix}
1 & 3 & 1
\\ a & 0 & 8
\\ -1 & a & -6
\end{pmatrix}
$\\
a) Determínese para qué valores de $a\in\mathbb{R}$ es invertible $A.$\\
b) Resuélvase para $a=0$ el sistema
\[
A\cdot \begin{pmatrix}
x\\y\\z
\end{pmatrix}=\begin{pmatrix}
0\\0\\0
\end{pmatrix}
\]
\end{ejer}

\begin{ejer}\em  Modelo 2016\\
Determínese la matriz $X$ que verifica $\begin{pmatrix}
3 & 1\\ -1 & 2
\end{pmatrix}\cdot X=\begin{pmatrix}
2 & 0\\1 & 4
\end{pmatrix}-\begin{pmatrix}
1 & 0\\ 4 & -1
\end{pmatrix}\cdot X.$
\end{ejer}

\begin{ejer}\em  Modelo 2016\\%
Se considera el sistema de ecuaciones lineales, dependiente del parámetro real  $a:$\\
\[
\left \{ \begin{matrix}
x+y-z=1
\\ 2x+2y-3z=3
\\ 3x+ay-2z=5
\end{matrix}\right.
\]
a) Discútase el sistema para los diferentes valores de $a.$\\
b) Resuélvase el sistema en el caso $a=2.$
\end{ejer}
